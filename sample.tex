\documentclass{fancyArticle}
\usepackage{lipsum}

\title{A Sample Document}
\author{Gary Baker}
\date{\today}

\begin{document}
\maketitle

\begin{abstract}
  \lipsum[1]
\end{abstract}

\section{A first section}

This\footnote{\lipsum[1][3]} demonstrates the footnote
style. \lipsum[4] Here's another footnote\footnote{\lipsum[30][1]}
followed by a numbered equation.

\begin{equation}
  \label{eq:1}
  e^{i\pi}+1 = 0
\end{equation}

\lipsum[5]

\subsection{A subsection}

\lipsum[6]

\begin{equation*} 
  V(l) = \max_{\beta\geq 0}\left\{ -c(\beta) * \Delta + e^{-r\Delta}
    \left[ \int \frac{1}{2} (\tanh(l/2)+1) V(l+\hat{s}) f_\beta(s) ds \right]\right\}
\end{equation*}

\lipsum[7]

\section{A section with some citations}

\cite{Nash1950} and \cite{Rubinstein1982} say some things about
bargaining. Those citations are also clickable links to their
respective entries in the bibliography.

\lipsum [20-21]

\pagebreak
\bibliography{sampleBib.bib}


\end{document}
